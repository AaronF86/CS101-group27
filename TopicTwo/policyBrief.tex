\documentclass{article}

\usepackage{biblatex}

\title{Policy Brief}
\author{CS101 Group 27}
\date{\today}

\begin{document}

\maketitle

\begin{abstract}
    The rise of online gig work platforms like Uber clearly exemplifies the 
    abuse of power within technology. This platform promotes a flexible and 
    independent work style for individuals who can't or wish to supplement 
    traditional employment with promises of time freedom. However many gig 
    workers face hidden costs such as a lack of stable income, minimal to no 
    workers' rights, and limited job security. Workers can be dismissed or 
    banned from platforms for minor infractions with no security. They are 
    often pressured to work long hours without benefits far exceeding 
    traditional job demands. Usually, gig platforms structure themselves as 
    marketplaces where freelancers bid on work, leading to undervaluing their 
    labour to secure jobs which can force them into producing low-quality 
    content. This content can be misused online to serve as a content mill for 
    their get-rich-quick scheme, by examining how existing frameworks and 
    labour laws fail to protect these workers and identifying key policymakers 
    and advocates. The target decision maker for Uber is Dara Khosrowshahi and 
    we aim to propose solutions that can offer these individuals the necessary 
    protections, such as providing ethical client safety so that the gig 
    workers feel safe when earning their wages for the day, offering training 
    opportunities to boost worker's salary and give them better job security 
    for the future. These solutions can ensure that gig workers can gain a 
    consistent income for themselves, while acknowledging that their protection 
    is valued within the company.
\end{abstract}

\end{document}