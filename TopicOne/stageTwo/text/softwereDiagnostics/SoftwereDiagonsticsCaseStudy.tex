\documentclass{article}
\usepackage{geometry}
\usepackage{graphicx}
\usepackage{hyperref}
\usepackage[style=authoryear,backend=biber]{biblatex} 

\addbibresource{references.bib} 

\title{Dark Data: Software Diagnostics Case Study}
\author{Aaron Fulton}
\date{\today}

\begin{document}

\maketitle

\begin{abstract}
This case study explores the concept of dark data for software diagnostics, such as software logging, to better understand the scale of dark data 
and the associated costs.
\end{abstract}

\section{Introduction}
According to \parencite{gartner2023darkdata}, dark data consists of information collected and stored during regular operations but that is rarely used for analysis 
or business value creation. This case study will examine data that fits this description within the context of software logging and other diagnostic tools. While 
these tools are designed to provide useful insights into software behavior much of the data remains unused as such becoming dark data.

\section{Why is this Data Considered Dark?}

Data is considered dark when it is not utilized for analytical purposes. When examining software logging as a source of dark data, we must account for the varying 
levels of expertise among users who interpret the logs. 

\subsection{Limited User Understanding}
End-users, for example are less likely to comprehend the information parsed in logs. As \parencite{David2020dartdata} notes, data is interpreted and analyzed by
 humans who apply their own understanding to make decisions based on that data. In the case of software logs users are often searching for specific errors or events 
 that have caused unexpected behavior, but interpreting these logs requires domain knowledge that many lack.

\subsection{The Therac-25 Case}
An example of the dangers of dark data in logs can be seen in the Therac-25 incident, where operators of a CNC radiation therapy machine failed to understand an 
error message leading to the injury or death of six patients. The vague error message, "Malfunction 54," was documented only as "dose input 2," which 
affected patient safety but was not clearly communicated \parencite{Leveson1993Therac25}. In this instance, the logging data was available but became dark due to 
its insufficient explanation and the lack of infomration for the operator.

as such software logging can be considered dark data when it is not fully utilized by its intended audience, leading to potential harm or missed opportunities for 
improvement.

\section{How Much Dark Data Exists?}
undstanding how much dark data is created from logging can be difficult because its darkness is determined by the user's ability to understand and make use of 
the data presented. However, we can attempt a rough estimate by examing a small sample of data and extrapolating from there.

\subsection{Data Collection and Extrapolation}
Based on a small survey from four games and two applications across four individuals, the average amount of logging data generated per program was found to be 
around 1132 MB. While games typically generate more comprehensive logs than applications, this number gives us a rough estimate for extrapolation.

The UK population is approximately 68.26 million \parencite{ons2022population}, and 90\% of households have access to a computer \parencite{ons2022technology}. With 
the average household size being 2.36 residents \parencite{ons2023householdsize}, the number of households can be calculated as follows:

\[
\mathrm{Households} = \frac{68.26 \mathrm{ \ million}}{2.36} \approx 28.92 \mathrm{ \ million \ households}.
\]

Of these, 90\% have computers:

\[
28.92 \cdot 0.9 \approx 26.03 \mathrm{ \ million \ households \ with \ computers}.
\]

Assuming each household generates logging data around 1132 MB per program, and an average PC has around 72 programs installed, the total amount of dark data can be
estimated:

\[
\mathrm{Dark \ Data} = 26.03 \cdot 72 \cdot 1132 \approx 2121.55\mathrm{ \ petabytes}.
\]

This rough estimate gives us a sense of the scale of dark data generated by logging, though the true figure may vary based on usage patterns and software complexity.

this data is colected from a small sample size of users from the games Ark: survival evolved, Minecraft, Space Engineers and War Thunder and the applications 
Chromium and Discord. these were chosen as they are commonly used applications and games by the group of individuals surveyed. this dats is in no way Conclusisve 
and is in fact a very small sample size. as this data was collected without care for Operating system or users time with the software that could lead to larger 
amounts of data being generated.

\section{What is the Cost of Dark Data?}
while the number we have calculated is the a very rought estimate we can use this to estimate the cost of dark data.

\subsection{Extrapolated Costs}
the cost of storaing data can very wildly depending on the scale and the method used (e.g ssd or hdd, m.2 or sata) but as we are looking at user data we can 
assume that the data will be storaged on user grade hardware using most system now run m.2 SSD that are mostly 1TB in size. the cost of a 1TB m.2 SSD is around 
\pounds100. knowing this we can calculate the cost of storing the data.

\[
\mathrm{Cost} = \frac{2121.55}{1024} \cdot 100 \approx \pounds207 \, \mathrm{million}.
\]

with a forther cost coming from power these drives with the most populer of these drives the Samsung 970 Evo Plus using 6.0W of power when active and as the 
average user will be using their computer for 4hr 23 mins a day \parencite{ukom2024online} we can calculate the cost of powering these drives. from the cost of power in the UK being around
\pounds0.24 per KWh we can calculate the cost of powering these drives. \parencite{ofgem2022powercost}
\[
\mathrm{Cost \ per \ day} = \frac{6.0 \, \mathrm{W}}{1000} \times 4.383 \, \mathrm{hours} \times 0.24 \, \frac{\pounds}{\mathrm{kWh}} \approx \pounds 0.0063 \, \mathrm{per \ day}.
\]

\[
\mathrm{Cost \ per \ year} = 0.0063 \times 365 \approx \pounds 2.30 \, \mathrm{per \ year}.
\]

To extrapolate this for 26.03 million PCs:

\[
\mathrm{Total \ cost \ per \ year} = 2.30 \times 26.03 \times 10^6 \approx \pounds 59.87 \, \mathrm{million \ per \ year}.
\]

\section{Solutions}
there are meny Solutions to this problem that can be used to currect this issue.

\subsection{Better Logging Practices}
One Solutions to create beter logging practices creating logs that are more human readable and that can probide mroe inforamtion of how to solve and fix issues
or that allow for data to be easily filter or searched for. as well as insuring that logs are removed after a set amount of time to reduce the amount of data
or when a threshold is reached.

\subsection{IT litracy}
another Solutions is to create programs to teach people how to understand and read these logs to allow for them to be used more effectivly.

\section{Conclusion}
In conclusion, dark data generated by software logging can be a issue, but this is meny Solutions to this problem that can and has been used to help fix this
issue. the cost of this data can be high but with the right tools and practices in place this cost can be reduced.

\printbibliography

\end{document}

